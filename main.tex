%% start of file `template.tex'.
%% Copyright 2006-2013 Xavier Danaux (xdanaux@gmail.com).
%
% This work may be distributed and/or modified under the
% conditions of the LaTeX Project Public License version 1.3c,
% available at http://www.latex-project.org/lppl/.


\documentclass[11pt,a4paper,sans]{moderncv}        % possible options include font size ('10pt', '11pt' and '12pt'), paper size ('a4paper', 'letterpaper', 'a5paper', 'legalpaper', 'executivepaper' and 'landscape') and font family ('sans' and 'roman')

% moderncv themes
\moderncvstyle{casual}                             % style options are 'casual' (default), 'classic', 'oldstyle' and 'banking'
\moderncvcolor{blue}                               % color options 'blue' (default), 'orange', 'green', 'red', 'purple', 'grey' and 'black'
%\renewcommand{\familydefault}{\sfdefault}         % to set the default font; use '\sfdefault' for the default sans serif font, '\rmdefault' for the default roman one, or any tex font name
%\nopagenumbers{}                                  % uncomment to suppress automatic page numbering for CVs longer than one page

% character encoding
\usepackage[utf8]{inputenc}                       % if you are not using xelatex ou lualatex, replace by the encoding you are using
%\usepackage{CJKutf8}                              % if you need to use CJK to typeset your resume in Chinese, Japanese or Korean

% adjust the page margins
\usepackage[scale=0.75]{geometry}
%\setlength{\hintscolumnwidth}{3cm}                % if you want to change the width of the column with the dates
%\setlength{\makecvtitlenamewidth}{10cm}           % for the 'classic' style, if you want to force the width allocated to your name and avoid line breaks. be careful though, the length is normally calculated to avoid any overlap with your personal info; use this at your own typographical risks...

% personal data
\name{Ivan}{Ogassavara}
\title{Cientista de Datos}                               % optional, remove / comment the line if not wanted
% \address{street and number}{postcode city}{country}% optional, remove / comment the line if not wanted; the "postcode city" and and "country" arguments can be omitted or provided empty
%\phone[mobile]{+1~(234)~567~890}                   % optional, remove / comment the line if not wanted
%\phone[fixed]{+2~(345)~678~901}                    % optional, remove / comment the line if not wanted
%\phone[fax]{+3~(456)~789~012}                      % optional, remove / comment the line if not wanted
\email{ivan.ogasawara@gmail.com}                               % optional, remove / comment the line if not wanted
\homepage{https://github.com/xmnlab/}                         % optional, remove / comment the line if not wanted
\extrainfo{http://lattes.cnpq.br/7764277601641080}                 % optional, remove / comment the line if not wanted
%\photo[64pt][0.4pt]{picture}                       % optional, remove / comment the line if not wanted; '64pt' is the height the picture must be resized to, 0.4pt is the thickness of the frame around it (put it to 0pt for no frame) and 'picture' is the name of the picture file
% \quote{Some quote}                                 % optional, remove / comment the line if not wanted

% to show numerical labels in the bibliography (default is to show no labels); only useful if you make citations in your resume
%\makeatletter
%\renewcommand*{\bibliographyitemlabel}{\@biblabel{\arabic{enumiv}}}
%\makeatother
%\renewcommand*{\bibliographyitemlabel}{[\arabic{enumiv}]}% CONSIDER REPLACING THE ABOVE BY THIS

% bibliography with mutiple entries
%\usepackage{multibib}
%\newcites{book,misc}{{Books},{Others}}
%----------------------------------------------------------------------------------
%            content
%----------------------------------------------------------------------------------
\begin{document}
%\begin{CJK*}{UTF8}{gbsn}                          % to typeset your resume in Chinese using CJK
%-----       resume       ---------------------------------------------------------
\makecvtitle

\section{Educación}
\cventry{2015--Interrumpida}{Posgrado en Ingeniería de Transportes y Gestión territorial}{Universidade Federal de Santa Catarina}{Florianópolis/Brasil}{\textit{Maestría}}{Las disciplinas abordaron una amplia linea de investigación y métodos para solucionar problemas de transportes.}  % arguments 3 to 6 can be left empty
\cventry{2004--2005}{Especialización en Sistemas de Información con énfasis en Tecnología de la Información}{Faculdade Eniac}{Guarulhos/Brasil}{\textit{Lato Sensu}}{Las disciplinas abordaron diversas tecnologías de información y comunicación. En la conclusión de la especialización fue entregada y defendida una monografía sobre La importancia del uso de software libre en telecentros comunitarios.}

\cventry{2003--2004}{Tecnología en Base de datos}{Faculdade Eniac}{Guarulhos/Brasil}{\textit{Pre-grado}}{Las clases ministradas tuvieron como enfoque el dominio de los conceptos sobre sistemas de base de datos, modelado de bases de datos y optimización}


\section{Experiencia}
\subsection{Laboral}
\cventry{2016--}{Cientista de datos y desarrollador de sistemas}{Independiente/Autónomo}{}{}{Análisis y visualización de datos; Investigación en literaturas científicas por métodos para solución de problemas; etc.\newline{}%
Logros detallados:%
\begin{itemize}%
\item Sistema de análisis de características visco-elásticas de pavimentos en ensayo en laboratorio;
\item Tablero de visualización de incidencia y estimación de influenza en Brasil;
\end{itemize}}
\cventry{year--year}{Job title}{Employer}{City}{}{Description line 1\newline{}Description line 2}
\subsection{Proyectos personales}
\cventry{year--year}{Job title}{Employer}{City}{}{Description}

\section{Languages}
\cvitemwithcomment{Portugués}{Nativo}{}
\cvitemwithcomment{Español}{Fluente}{}
\cvitemwithcomment{Inglés}{Intermedio}{Buena lectura, buena comprensión auditiva, escritura técnica}
\cvitemwithcomment{Italiano}{Intermedio}{Buena lectura, buena comprensión auditiva}

%\section{Computer skills}
%\cvdoubleitem{category 1}{XXX, YYY, ZZZ}{category 4}{XXX, YYY, ZZZ}
%\cvdoubleitem{category 2}{XXX, YYY, ZZZ}{category 5}{XXX, YYY, ZZZ}
%\cvdoubleitem{category 3}{XXX, YYY, ZZZ}{category 6}{XXX, YYY, ZZZ}

%\section{Interests}
%\cvitem{hobby 1}{Description}
%\cvitem{hobby 2}{Description}
%\cvitem{hobby 3}{Description}

\section{Cursos extras}
\cvlistitem{Intro to Hadoop and MapReduce. Udacity (2016)}
\cvlistitem{Intro to Data Science. Udacity (2014)}
\cvlistitem{Machine Learning: Supervised Learning. Udacity (2014)}
\cvlistitem{Intro to Descriptive Statistics. Udacity (2014)}
\cvlistitem{Design of Computer Programs. Udacity (2012)}

%\section{Extra 2}
%\cvlistdoubleitem{Item 1}{Item 4}
%\cvlistdoubleitem{Item 2}{Item 5\cite{book1}}
%\cvlistdoubleitem{Item 3}{Item 6. Like item 3 in the single column list before, this item is particularly long to wrap over several lines.}

%\section{References}
%\begin{cvcolumns}
%  \cvcolumn{Category 1}{\begin{itemize}\item Person 1\item Person 2\item Person 3\end{itemize}}
%  \cvcolumn{Category 2}{Amongst others:\begin{itemize}\item Person 1, and\item Person 2\end{itemize}(more upon request)}
%  \cvcolumn[0.5]{All the rest \& some more}{\textit{That} person, and \textbf{those} also (all available upon request).}
%\end{cvcolumns}

% Publications from a BibTeX file without multibib
%  for numerical labels: \renewcommand{\bibliographyitemlabel}{\@biblabel{\arabic{enumiv}}}% CONSIDER MERGING WITH PREAMBLE PART
%  to redefine the heading string ("Publications"): \renewcommand{\refname}{Articles}
\nocite{*}
\bibliographystyle{plain}
\bibliography{publications}                        % 'publications' is the name of a BibTeX file

% Publications from a BibTeX file using the multibib package
%\section{Publications}
%\nocitebook{book1,book2}
%\bibliographystylebook{plain}
%\bibliographybook{publications}                   % 'publications' is the name of a BibTeX file
%\nocitemisc{misc1,misc2,misc3}
%\bibliographystylemisc{plain}
%\bibliographymisc{publications}                   % 'publications' is the name of a BibTeX file

\end{document}


%% end of file `template.tex'.
