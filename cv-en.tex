\documentclass[10pt,a4paper]{article}
\usepackage{template}

% Footer contact
\SetFooter{+591 69757089}{me@ivanogasawara.com}{La Paz, Bolivia}

\begin{document}

\StartCVColumns

% -------- LEFT COLUMN --------
% (Add your photo if you want)
% \Photo[4.0cm]{photo.jpg}

\LeftSection{My Profile}
\small
Research Software Engineer and tech leader with 22+ years in software engineering. Across my career I've delivered backend systems, data platforms, DevOps/packaging workflows, and, more recently, ML/RAG and LLVM/compilers work.

I contribute by designing robust systems, applying best practices, improving team collaboration, adopting open-source technologies, and authoring libraries and tools. Polyglot background: I've contributed to projects using Python, C/C++, JavaScript, Java, R, Rust, Bash, PHP, VB6, ASP, ActionScript, etc., and I'm comfortable learning new languages and stacks as needed.

Founder of Open Science Labs; mentor for internships and Google Summer of Code; Advisor at pyOpenSci; Steering Committee member, Open Science Framework - Open Source Community.
\normalsize


\LeftSection{Skills}
\small
\textbf{Hard skills}
\begin{tightitemize}
  \item Programming: Arx, Python, C/C++, JS; also Java, R, Rust, Bash, PHP, VB6, ASP, ActionScript.
  \item Backend \& APIs: Django, DRF, Flask; task queues (Celery); packaging \& release workflows.
  \item Data \& ML: pandas, scikit-learn; RAG; visualization (Plotly/Bokeh).
  \item DevOps: Docker; CI/CD (GitHub Actions); Makim, Sugar, Ansible.
  \item Databases \& Search: PostgreSQL; Elasticsearch.
  \item Compilers: LLVM; ASTx, IRx, Arx.
\end{tightitemize}

\textbf{Soft skills}
\begin{tightitemize}
  \item Technical leadership; team coaching; code review \& standards.
  \item Mentoring (internships, Google Summer of Code).
  \item Architecture \& strategy; documentation; reproducibility practices.
  \item Community building; open-source governance (pyOpenSci Advisor; OSF Open Source Community Steering Committee).
  \item Workshop facilitation \& teaching assistant.
\end{tightitemize}

\textbf{Languages}
\begin{tightitemize}
  \item Portuguese (Native)
  \item Spanish (Fluent)
  \item English (Advanced)
  \item Italian (Basic)
\end{tightitemize}

\RightSection{Teaching, Training \& Workshops}
\begin{tightitemize}
  \item Mentoring: Open Source projects, Compilers, DevOps, Health Care systems, packaging, CI/CD.
  \item Training: Python \& data science foundations.
  \item Programs: Internship \& GSoC mentorship;
  \item Events: Conference organization and facilitation.
\end{tightitemize}

\normalsize

% -------- RIGHT COLUMN --------
\switchcolumn

\NameBlock{Ivan Ogasawara}{Research Software Engineer, Tech Leader,  Executive Lead}

\RightSection{Experience}

\ExpHead{Teaching Assistant}{WorldQuant University — Applied Data Science Lab (Remote)}{2024--Present}
\begin{tightitemize}
  \item Support learners in WQU's Applied Data Science Lab.
  \item Coach students through \emph{eight} end-to-end projects covering data wrangling, visualization, regression/ML pipelines, time series, and database/API access; promote collaboration and use of live office hours.
  \item Guide Python/pandas and scikit-learn workflows; reinforce SQL/MongoDB/API data access, model evaluation, and clear communication of results.
\end{tightitemize}

\ExpHead{Instructor — Python Basics \& Beyond: Intro to Data Analysis with Python}{The GRAPH Courses (Remote)}{2024--Present}
\begin{tightitemize}
  \item Teach a live bootcamp covering Python foundations, pandas, Plotly, Quarto dashboards, and Git/GitHub.
  \item Lead weekly workshops and code reviews; mentor projects using industry/research datasets and support capstone/portfolio work.
\end{tightitemize}

\ExpHead{Teaching Assistant — Generative AI for Work \& Research}{The GRAPH Courses (Remote)}{2024--Present}
\begin{tightitemize}
  \item Support an 8-week, hands-on program on no-code AI tools to automate content, research, data analysis, and knowledge-management workflows.
  \item Facilitate workshops/help sessions on practical tool use (docs/spreadsheets/data automation; literature tools; custom chatbots/RAG); coach capstone projects.
\end{tightitemize}

\ExpHead{Founder \& Executive Director}{Open Science Labs}{2015--Present}
\begin{tightitemize}
  \item Built a supportive global community with study groups, resources, events, and contributor pathways.
  \item Launched Project Incubator \& Affiliation program; mentored interns and GSoC contributors across several countries.
\end{tightitemize}

\ExpHead{Research Software Engineer (Tech \& Team Lead)}{The GRAPH Network / LiteRev}{2023--Present}
\begin{tightitemize}
  \item Literature-review platform: Python, Django, scikit-learn, pacmap, optuna, hdbscan, bokeh, Celery, Docker, CI/CD.
\end{tightitemize}

\ExpHead{Software Developer}{Quansight / OSBIG}{2018--2021}
\begin{tightitemize}
  \item Contributions to PyTorch, Ibis-framework, Jupyter, OmniSciDB; client work in Django/DRF, asyncio, Docker, GraphQL, Argo.
\end{tightitemize}

\ExpHead{Software Packager}{Anaconda, Inc. (Contract)}{2017}
\begin{tightitemize}
  \item Conda packaging for Python, R, and Java libraries; reproducible builds and dependency pinning.
\end{tightitemize}

\ExpHead{Research Software Engineer}{Fundação para o Desenvimento Científico e Tecnológico em Saúde — Fiotec (Contract)}{2016--2017}
\begin{tightitemize}
  \item Built portals for visualization and estimation of influenza and dengue incidence in Brazil.
  \item Stack: Python, JavaScript, pandas, Bokeh, Django/Flask, Fiona, Rasterio, PostgreSQL, MapServer, Docker.
\end{tightitemize}

\ExpHead{Research Software Engineer}{Fapeu — Fundação de Amparo a Pesquisa e Extensão Universitária (Contract)}{2016}
\begin{tightitemize}
  \item Web system for laboratory analysis of visco-elastic pavement characteristics.
\end{tightitemize}

\ExpHead{Research Software Engineer}{Fapeu — Fundação de Amparo a Pesquisa e Extensão Universitária (Florianópolis, Brazil · Full-time)}{2013--2016}
\begin{tightitemize}
  \item Literature analysis in transportation engineering (WIM, fatigue, complex modulus).
  \item Implemented DSP/ML/HPC workflows for data processing and modeling.
\end{tightitemize}

\RightSection{Education}
\EduEntry{Transportation Engineering — Master's (interrupted)}{Universidade Federal de Santa Catarina (UFSC)}{2015--2016}
\EduEntry{Information Technology — Post-Grad (Lato Sensu)}{Faculdade Eniac}{2004--2005}
\EduEntry{Database Technology — Associate's Degree}{Faculdade Eniac}{2003--2004}

\EndCVColumns
\end{document}
