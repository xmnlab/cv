\documentclass[11pt,a4paper]{article}
\usepackage{coverletter}

\begin{document}

\CLHeaderMinimal
  {Ivan Ogasawara}
  {}
  {\href{mailto:me@ivanogasawara.com}{me@ivanogasawara.com} · +591 69757089 · \href{https://www.linkedin.com/in/ivan-ogasawara/}{linkedin.com/in/ivan-ogasawara}}

\CLRecipient
  {Hiring Committee — GESIS - Leibniz Institute for the Social Sciences}
  {Köln / Mannheim, Germany}
  {\today}

\CLSubject{Application: Research Software Engineer (SSE-07) — ScaleRep}

Dear Hiring Committee,

I am writing to apply for the position of \emph{Research Software
Engineer (SSE-07)} in the Service Strategy \& Engineering department,
contributing to the ScaleRep project on multilingual answer scales. As
a Senior Research Software Engineer and technical lead with over
22 years of software engineering experience, including research contexts, I
have built and operated platforms that improve data quality,
reproducibility, and access for scientific communities.

In my current roles at The GRAPH Network, I lead backend and DevOps
work for \emph{LiteRev}, a literature-review platform, and previously
for \emph{EpiGraphHub}, a data-centric epidemiological analysis
platform. There, I design and implement services using Python, Django,
task queues (Celery), and REST APIs, backed by PostgreSQL and
Elasticsearch, and deploy them with Docker, CI/CD (GitHub Actions),
Makim, Sugar, and Ansible. I have also packaged Python, R, and Java
libraries at Anaconda, Inc., ensuring reproducible builds and careful
dependency management.

Earlier, as a Research Software Engineer at Fiotec and Fapeu, I
developed data portals and analysis systems for influenza, dengue, and
transportation engineering. These projects required close collaboration
with domain experts, translation of methodological requirements into
robust software, and technical data preparation for analysis and
visualization. This experience maps naturally onto requirement
gathering, technical data preparation, and iterative refinement.

My background includes programming in Java (with few projects), JavaScript, and Vue.js
alongside Python, C/C++, and other languages, including the creation of my own compiler.
This gives me a solid foundation to work in service-oriented architectures and to become
productive with Angular-based frontends while collaborating with UX
specialists and conducting user tests. I am used to designing and
evolving services that must be reliable, maintainable, and well
documented for research users.

In parallel, I am deeply engaged in training and community work. As a
Teaching Assistant at WorldQuant University's Applied Data Science Lab
and Instructor at The GRAPH Courses, I support
learners through end-to-end data projects, lead workshops and code
reviews, and help participants build clear, reproducible workflows. As
Founder and Executive Director of Open Science Labs, I mentor
contributors and Google Summer of Code participants, and I have served
as an Advisor at pyOpenSci and on the Steering Committee of the Open
Science Framework - Open Source Community. These roles reflect my
commitment to building sustainable research software and supporting
diverse user communities.

I am fluent in Portuguese, Spanish, and English and have basic Italian,
which, combined with my experience in international research
collaborations, fits well with the multilingual and cross-cultural
focus of ScaleRep and the international environment at GESIS.

I would be delighted to contribute my research software engineering,
DevOps, and training experience to ScaleRep and to GESIS more broadly.
Thank you very much for considering my application.

\CLSignoff{Sincerely}{Ivan Ogasawara}{}

\end{document}
